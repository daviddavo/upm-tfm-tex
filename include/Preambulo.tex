% -*-coding: utf-8 -*-
%%***********************************************
%% Plantilla para TFG.
%% Escuela Técnica Superior de Ingenieros Informáticos. UPM.
%%***********************************************
%% Preámbulo del documento.
%%***********************************************
\documentclass[a4paper,11pt,twoside]{book}
\usepackage[utf8]{inputenc}
\usepackage[T1]{fontenc}
\usepackage[english,spanish,es-lcroman,es-tabla]{babel}
\usepackage{bookman}
\decimalpoint
\usepackage{graphicx}
\usepackage{amsfonts,amsgen,amsmath,amssymb}
\usepackage[top=3cm, bottom=3cm, right=2.54cm, left=2.54cm]{geometry}
\usepackage{afterpage}
\usepackage{colortbl,longtable}
\usepackage{pdfpages}
\usepackage[hyphens]{url}
\usepackage[pdfborder={0 0 0}]{hyperref} 
\usepackage[stable]{footmisc}
\usepackage{parskip} % para separar párrafos con espacio.
%%-----------------------------------------------
% No estaban en la plantilla originalmente
\usepackage{multicol,multirow}
\usepackage{wasysym}
\usepackage{etoolbox}
\usepackage[inline]{enumitem}
\usepackage[acronym,toc,shortcuts]{glossaries}
\makeglossaries
\usepackage[
    sorting=none,
    style=numeric,
    defernumbers=true,
]{biblatex}
\addbibresource{include/all-references.bib}
\usepackage{csquotes}
\usepackage{subcaption}
\usepackage{wrapfig}
\usepackage{sidecap}
% \usepackage{minted}
% \setminted{
%     frame=lines,
%     framesep=2mm,
%     %  bgcolor=LightGray,
%     fontsize=\footnotesize,
%     linenos,
% }
\usepackage{float}
\newfloat{code}{thp}{lop}[chapter]
\floatname{code}{Código}
\usepackage{booktabs}
\usepackage{threeparttable}
\usepackage{datetime2}
\usepackage{siunitx}

\DeclareBibliographyCategory{cited}
\AtEveryCitekey{\addtocategory{cited}{\thefield{entrykey}}}

% TODO: Remove before submission
% \usepackage[left,pagewise]{lineno}
% \linenumbers
%%-----------------------------------------------
\usepackage{fancyhdr}
\pagestyle{fancy}
\fancyhf{}
\fancyhead[LO]{\leftmark}
\fancyhead[RE]{\rightmark}

% TODO: Remove before submission
% \fancyfoot[R]{\footnotesize Versión \DTMnow}

\setlength{\headheight}{1.5\headheight}
\cfoot{\thepage}

\addto\captionsspanish{ \renewcommand{\contentsname}
  {Tabla de contenidos} }
\setcounter{tocdepth}{4}
\setcounter{secnumdepth}{4}

\renewcommand{\chaptermark}[1]{\markboth{\textbf{#1}}{}}
\renewcommand{\sectionmark}[1]{\markright{\textbf{\thesection. #1}}}
\newcommand{\HRule}{\rule{\linewidth}{0.5mm}}
\newcommand{\bigrule}{\titlerule[0.5mm]}

\usepackage{appendix}
\renewcommand{\appendixname}{Anexos}
\renewcommand{\appendixtocname}{Anexos}
%\renewcommand{\appendixpagename}{Anexos}
\DeclareCaptionFormat{source}{%
  \ifx\captionsource\relax\relax\else
    \captionsource\par
  \fi
  #1#2#3\par}
\newcommand*\setcaptionsource[1]{\def\captionsource{{\footnotesize\textit{Fuente:}~#1}}}
\let\captionsource\relax
\captionsetup{format=source}

\setcounter{secnumdepth}{3} % Niveles numerados hasta subparagraph
\setcounter{tocdepth}{3}    % Niveles incluidos en el índice hasta subparagraph
%%-----------------------------------------------
%% Páginas en blanco sin cabecera:
%%-----------------------------------------------
\usepackage{dcolumn}
\newcolumntype{.}{D{.}{\esperiod}{-1}}
\makeatletter
\addto\shorthandsspanish{\let\esperiod\es@period@code}

\def\clearpage{
  \ifvmode
    \ifnum \@dbltopnum =\m@ne
      \ifdim \pagetotal <\topskip
        \hbox{}
      \fi
    \fi
  \fi
  \newpage
  \thispagestyle{empty}
  \write\m@ne{}
  \vbox{}
  \penalty -\@Mi
}
\makeatother
%%-----------------------------------------------
%% Estilos código de lenguajes: Consola, C, C++ y Python
%%-----------------------------------------------
\usepackage{color}

\definecolor{gray97}{gray}{.97}
\definecolor{gray75}{gray}{.75}
\definecolor{gray45}{gray}{.45}

\usepackage{listings}
\lstset{ frame=Ltb,
     framerule=0pt,
     aboveskip=0.5cm,
     framextopmargin=3pt,
     framexbottommargin=3pt,
     framexleftmargin=0.4cm,
     framesep=0pt,
     rulesep=.4pt,
     backgroundcolor=\color{gray97},
     rulesepcolor=\color{black},
     %
     stringstyle=\ttfamily,
     showstringspaces = false,
     basicstyle=\scriptsize\ttfamily,
     commentstyle=\color{gray45},
     keywordstyle=\bfseries,
     %
     numbers=left,
     numbersep=6pt,
     numberstyle=\tiny,
     numberfirstline = false,
     breaklines=true,
   }
\lstnewenvironment{listing}[1][]
   {\lstset{#1}\pagebreak[0]}{\pagebreak[0]}

\lstdefinestyle{consola}
   {basicstyle=\scriptsize\bf\ttfamily,
    backgroundcolor=\color{gray75},    
    }

\lstdefinestyle{CodigoC}
   {basicstyle=\scriptsize,
	frame=single,
	language=C,
	numbers=left
   }
   
\lstdefinestyle{CodigoC++}
   {basicstyle=\small,
	frame=single,
	backgroundcolor=\color{gray75},
	language=C++,
	numbers=left
   }

\lstdefinestyle{Python}
   {language=Python,    
   }
\makeatother   