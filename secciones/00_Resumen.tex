\chapter*{Resumen}

Las Organizaciones Autónomas Descentralizadas (\textit{Decentralized Autonomous Organizations, DAOs}) han surgido como un nuevo enfoque hacia la gobernanza colectiva, facilitado por la tecnología blockchain. Las DAOs fomentan procesos de votación democráticos, permitiendo a los miembros proponer y votar sobre propuestas, moldeando así colectivamente el futuro de la organización. Sin embargo, la efectividad y legitimidad de la toma de decisiones dentro de las DAOs puede verse afectada por la baja participación de los votantes, un desafío común con otras comunidades en línea y los sistemas de votación tradicionales. Dada el tiempo limitado de votación de cada propuesta, las técnicas convencionales en sistemas recomendadores son inadecuadas. Por esa razón, este Trabajo de Fin de Máster introduce el primer sistema de recomendación diseñado específicamente para DAOs. En particular, se han desarrollado nuevas técnicas de validación y una nueva linea base. Este enfoque ha sido probado en tres modelos: un primero basado en filtrado colaborativo que utiliza Redes Neuronales de Grafos para explotar el grafo bipartito miembro-propuesta formado por la votación en DAOs; un segundo basado en contenido que utilizando Procesamiento del Lenguaje Natural; y un tercer modelo híbrido que combina los resultados de los dos anteriores. Aunque el proyecto se ha realizado con la organización de Decentraland en mente, que gobierna colectivamente una plataforma de metaverso con más de 35~000 miembros y 2~000 propuestas votadas, también se ha probado y comparado con otras organizaciones con diferentes características. Estos resultados no solo demuestran el potencial de los sistemas de recomendación para mejorar la personalización de propuestas y mejorar la participación de los votantes en las DAOs, sino que también se alinean con las mejoras observadas en la participación de los usuarios en otros proyectos colaborativos en línea que han implementado sistemas similares. Además, el sistema de recomendación basado en GNN con restricciones temporales podría ser adaptado a contextos similares, como la recomendación de eventos.

\textbf{Palabras clave}: Organizaciones Autónomas Descentralizadas, Sistemas Recomendadores, Redes Neuronales de Grafos, Procesamiento de Lenguaje Natural, Blockchain

%%--------------
\newpage
%%--------------

\chapter*{Abstract}
\begin{otherlanguage}{english}
Decentralized Autonomous Organizations (DAOs) have emerged as a novel approach to collective governance, facilitated by blockchain technology. DAOs foster democratic voting processes, allowing members to put forward and vote on proposals, thereby collectively shaping the organization’s future. However, the effectiveness and legitimacy of decision-making within DAOs can be compromised by low voter turnout, a challenge shared with traditional online communities and voting systems. Given the limited lifespan of each proposal, conventional recommender system techniques are unsuitable. In response to these issues, this Master's Thesis introduces a recommender system specifically designed for DAOs. In particular, new validation techniques and a baseline had to be developed. The approach has been tested on three models, a model that leverages Graph Neural Networks (GNN) for collaborative filtering, effectively exploiting the member-proposal bipartite graph inherent in DAOs voting, a second one that utilizes Natural Language Processing as a content-based approach, and a third hybrid model that combines the results of the previous two. While the project has been made with the Decentraland DAO organization in mind, which collectively governs a metaverse platform with over 35,000 members and 2,000 proposals voted, it has also been tested on and compared with other organizations with different characteristics. We compare our approach with a baseline that recommends the most popular open proposals at the time of recommendation. These models accurately predict future voters, surpassing the proposed baseline. These results not only underscore the potential of recommender systems in enhancing voter participation within DAOs but also align with the observed improvements in user engagement in other online collaborative projects that have implemented similar systems. Furthermore, our GNN-based recommendation systems with temporal constraints could be adapted to other settings such as event recommendation.

\textbf{Keywords}: Decentralized Autonomous Organizations, Recommender Systems, Graph Neural Networks, Natural Language Processing, Blockchain
\end{otherlanguage}


%%%%%%%%%%%%%%%%%%%%%%%%%%%%%%%%%%%%%%%%%%%%%%%%%%%%%%%%%%%
%% Final del resumen. 
%%%%%%%%%%%%%%%%%%%%%%%%%%%%%%%%%%%%%%%%%%%%%%%%%%%%%%%%%%%